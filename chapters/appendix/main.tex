\chapter{The Boosted Frame Solver in PyECLOUD}
\label{ch:BoostedFrameCode}
We report the implementation of the boosted frame in PyECLOUD as we explained it in CHAPTER. We implemented a \textsc{space\_charge\_electromagnetic} class which is an extension of the pre-existing \textsc{space\_charge\_class} containing the electrostatic solver. The new class implements the functions \textsc{recompute\_spchg\_emfield} and \textsc{get\_sc\_em\_field}. In the former the sources are scattered from the particles to the meshes nodes and the potentials are computed using the interface to the PyPIC solvers, which is a self-standing package containing the Poisson solver (\href{https://github.com/PyCOMPLETE/PyPIC}{https://github.com/PyCOMPLETE/PyPIC}). In the latter we compute the electromagnetic field from the potentials and we gather it at the position of the macroparticles.\\
The full PyECLOUD code is publicly available on Github in the repository \href{https://github.com/PyCOMPLETE/PyECLOUD}{https://github.com/PyCOMPLETE/PyECLOUD}.

\begin{lstlisting}[language=Python, basicstyle=\small,]
import numpy as np
from .space_charge_class import space_charge
from scipy.constants import epsilon_0, mu_0
from scipy.constants import c  as c_light

from . import int_field_for as iff
import sys
if sys.version_info.major>2:
    from io import StringIO
else:
    from io import BytesIO as StringIO

import matplotlib.pyplot as plt
import matplotlib as mpl
from . import rhocompute as rhocom
from mpl_toolkits.axes_grid1 import make_axes_locatable
from matplotlib import colors
from . import int_field_for as iff

na = lambda x: np.array([x])

class space_charge_electromagnetic(space_charge, object):

    def __init__(self, chamb, Dh, gamma, Dt_sc=None, 
                 sparse_solver='scipy_slu',
                 f_telescope=None, target_grid=None, 
                 N_nodes_discard=None, N_min_Dh_main=None,
                 Dh_U_eV=None,
            PyPICmode='FiniteDifferences_ShortleyWeller'):

        super(space_charge_electromagnetic, 
              self).__init__(chamb, Dh, Dt_sc, 
                             PyPICmode , sparse_solver, 
                             f_telescope, target_grid, 
                             N_nodes_discard, 
                             N_min_Dh_main, Dh_U_eV)

        self.flag_em_tracking = True
        # Initialize additional states for vector potential
        # The text trap reduces the amount of output shown 
        # in the terminal
        text_trap = StringIO()
        sys.stdout = text_trap
        self.state_Ax = self.PyPICobj.get_state_object()
        self.state_Ay = self.PyPICobj.get_state_object()
        sys.stdout = sys.__stdout__
        self.state_Ax_old = None
        self.state_Ay_old = None

        # Store the relativistic factors of the beam
        self.gamma = gamma
        self.beta = np.sqrt(1-1/(gamma*gamma))

    def recompute_spchg_emfield(self, MP_e, 
                                flag_solve=True,
                                flag_reset=True):
        # Update the old states before scattering
        # The text trap reduces the amount of output shown 
        # in the terminal
        text_trap = StringIO()
        sys.stdout = text_trap
        self.state_Ax_old = self.state_Ax.get_state_object()
        self.state_Ay_old = self.state_Ay.get_state_object()
        sys.stdout = sys.__stdout__

        # Scatter the charge
        self.PyPICobj.scatter(MP_e.x_mp[0:MP_e.N_mp],
                              MP_e.y_mp[0:MP_e.N_mp], 
                              MP_e.nel_mp[0:MP_e.N_mp],
                              charge=MP_e.charge, 
                              flag_add=not(flag_reset))

        # Scatter currents
        inv_c2 = 1/(epsilon_0 * mu_0)
        Jx_mp = (MP_e.nel_mp[0:MP_e.N_mp] * 
                 MP_e.vx_mp[0:MP_e.N_mp])
        self.state_Ax.scatter(MP_e.x_mp[0:MP_e.N_mp], 
                              MP_e.y_mp[0:MP_e.N_mp],
                              inv_c2 * Jx_mp[0:MP_e.N_mp],
                              charge=MP_e.charge, 
                              flag_add=not(flag_reset))

        Jy_mp = (MP_e.nel_mp[0:MP_e.N_mp] * 
                 MP_e.vy_mp[0:MP_e.N_mp])
        self.state_Ay.scatter(MP_e.x_mp[0:MP_e.N_mp], 
                              MP_e.y_mp[0:MP_e.N_mp],
                              inv_c2 * Jx_mp[0:MP_e.N_mp],
                              charge=MP_e.charge, 
                              flag_add=not(flag_reset))

        # Solve the Poisson problems
        if flag_solve:
            self.PyPICobj.solve()
            self.PyPICobj.solve_states([self.state_Ax, 
                                        self.state_Ay])

    def get_sc_em_field(self, MP_e):
        # Compute un-primed potentials (with wrong sign 
        # becase gather is meant to return E field..)
        Ax_obj = self.state_Ax
        Ay_obj = self.state_Ay
        _, m_dAx_dy = Ax_obj.gather(MP_e.x_mp[0:MP_e.N_mp],
                                 MP_e.y_mp[0:MP_e.N_mp])
        m_dAy_dx, _ = Ay_obj.gather(MP_e.x_mp[0:MP_e.N_mp],
                                 MP_e.y_mp[0:MP_e.N_mp])
        
        phi_obj = self.PyPICobj
        (m_dphi_dx, 
         m_dphi_dy) = phi_obj.gather(MP_e.x_mp[0:MP_e.N_mp],
                                     MP_e.y_mp[0:MP_e.N_mp])
        # Fix signs
        dAx_dy = -m_dAx_dy
        dAy_dx = -m_dAy_dx

        Ax_obj_old = self.state_Ax_old
        Ay_obj_old = self.state_Ay_old
        # If not first passage compute time derivatives 
        # of Ax and Ay
        if (Ax_obj_old != None and 
            Ay_obj_old != None):
            dAx_dt = (Ax_obj.gather_phi(MP_e.x_mp[0:MP_e.N_mp],
                                        MP_e.y_mp[0:MP_e.N_mp]) -
                      Ax_obj_old.gather_phi(MP_e.x_mp[0:MP_e.N_mp],
                              MP_e.y_mp[0:MP_e.N_mp]))/self.Dt_sc
            dAy_dt = (Ay_obj.gather_phi(MP_e.x_mp[0:MP_e.N_mp],
                                        MP_e.y_mp[0:MP_e.N_mp]) -
                      Ay_obj_old.gather_phi(MP_e.x_mp[0:MP_e.N_mp],
                              MP_e.y_mp[0:MP_e.N_mp]))/self.Dt_sc

        # If first passage set derivatives to zero
        else:
            dAx_dt = np.zeros(MP_e.N_mp)
            dAy_dt = np.zeros(MP_e.N_mp)

        # Compute E-field in lab frame (E = -grad phi - dA/dt)
        Ex_sc_n = m_dphi_dx - dAx_dt
        Ey_sc_n = m_dphi_dy - dAy_dt

        # Compute B-field in lab frame (B = curl A)
        Bx_sc_n = 1/(self.beta*c_light)*dAy_dt
        By_sc_n = -1/(self.beta*c_light)*dAx_dt
        Bz_sc_n = dAy_dx - dAx_dy

        return Ex_sc_n, Ey_sc_n, Bx_sc_n, By_sc_n, Bz_sc_n

\end{lstlisting}

All the studies from the first chapter are stored in the repository TODO.

